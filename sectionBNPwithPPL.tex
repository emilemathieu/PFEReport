
\chapter{BNP sampling in PPL}

Stochastic Memoization with DPmem: $\alpha = 0$, deterministic memoization, $\alpha = \inf$ no memoization\\
https://probmods.org/chapters/12-non-parametric-models.html


\section{Link between BNP and High order PPL}
See Frank Wood meeting
Frank's Theorem: "The Lambda abstraction is necessary in a PPL to be able to denote BNP in this language." The Lambda abstraction seems to be synonymous to function recursion and thus allowing stochastic loop, which are useful for BNP.

Stick-breaking process = control flow + recursion = stochastic recursion ?

Talk lambda abstraction
discrete probability with countable support -> cannot represent it in memory
-> lazy recursive representation the measure
-> Urn: posterior predictive distribution of $X_{k+1}$ given $X_0, \dots, X_{k}$
CRP: simple sufficient statistics, but in general not nice like that
All marginal samplers: Urn distributions, involve complicated integral (can sometimes introduce auxiliary variables)

Stick-breaking process: construction to sample a size-biased permutation of the random probability measure

\subsection{Implementation}
In Clojure there is a recursive operator for tail-recursive functions that uses constant space. Without such a specific operator, we will have a Maximum call stack size exceeded error for heavy tail BNP such as for some PY stick-breaking process. Unfortunately, \gls{TCO} is not implemented in Julia (since it's not implemented in LLVM on which Julia relies). But other solutions in Julia exist to avoid stack overflow.

Clojure provides special forms loop and recur for writing tail-recursive programs. Anglican programs are \gls{CPS}-onverted and do not use the stack; recursive calls in Anglican cannot lead to stack overflow.

A function is tail-recursive when the recursive call happens as the final action in a function, in which case it can happen without the function call stack growing. In continuation-passing style, there is no stack – all function calls are tail calls, hence all recursive functions are tail-recursive.

\section{Stick-breaking process}
PPL -> be able to sample from prior

\subsection{DP and PY}
well known
already implemented in most \gls{PPL}s

\subsection{PK}
Theorem 2.1 of Perman et al. states that the sequence of surplus masses $(T_k)_(k>=0)$ forms a Markov chain and gives the corresponding initial distribution and transition kernels.

Could we sample the Kth stick length in generic way since we know the density ? For instance by restricting to Levy measure intensity and Total mass density which are tractable (not σ-stable PK). One may use a simple rejection sampling with proposal $U(0, t_K)$ (what would be M ?).

There exist other models based directly on specifying a valid distribution for the stick breaks (e.g. Ottowa sequence, Beta-Stacy, others?); these should fit in without a problem.

\subsection{PG}
cf article Ben

\section{Sampler}

highly dimensional -> particle methods
write model as streaming/online to avoid IS and take advantage of resampling

prior on parameters -> PMCMC
PG -> path degeneracy
PPMH -> ?
PGAS / IPMCMCM

\section{Theoretical aspects}
calculus for SMC for BNP mixture with fixed parameters
fixed parameters are not a assumption, since can be made random then with PMCMC

\section{Open questions}
Is a stick-breaking-like Markov Chain necessary and sufficient for doing inference with particle methods? For lazy instantiation (possibly equivalent to denotable / lambda abstraction?)?

What should be the representation of the state in the PPL (theoretically and efficiently concerned) ? Unique components + assignments | all components | sticks lengths ?   
