
\chapter{BNP inference within PPL} \label{BNP_PPL}
For now we have restricted ourselves to infinite mixture models, but we hope that our framework will enable efficient sampling for other \gls{BNP} models such as the infinite \acrlong{HMM}. We have also focused our study on sampling methods, but \acrlong{VI} is considered in Section \ref{BNP_VI}.

\textcolor{red}{Stochastic Memoization perspective ??}
An interesting perspective of Discrete Random Probability Measure is through \textit{stochastic memoization} \footnote{\url{https://probmods.org/chapters/12-non-parametric-models.html}}.
First let's recall what is the usual (deterministic) memoization.
For instance with the \gls{DP}, with $\alpha = 0$ we recover a deterministic memoization whereas with $\alpha \rightarrow \inf$ there is no memoization at all.


\section{Link between BNP and High order PPL / Recursion}
% Bayesian prob/BNP fits naturally the framework of PPL

See Frank Wood meeting
Frank's Theorem: "The Lambda abstraction is necessary in a PPL to be able to denote BNP in this language." The Lambda abstraction seems to be synonymous to function recursion and thus allowing stochastic loop, which are useful for BNP.

Stick-breaking process = control flow + recursion = stochastic recursion ?

Talk lambda abstraction
discrete probability with countable support -> cannot represent it in memory
-> lazy recursive representation the measure
-> Urn: posterior predictive distribution of $X_{k+1}$ given $X_0, \dots, X_{k}$
CRP: simple sufficient statistics, but in general not nice like that
All marginal samplers: Urn distributions, involve complicated integral (can sometimes introduce auxiliary variables)

Stick-breaking process: construction to sample a size-biased permutation of the random probability measure

For lazy instantiation (possibly equivalent to denotable / lambda abstraction?)?

want to sample from a discrete distribution with countable support
->finite memory
to represent it, needs a recursive representation

\subsection{Existing}
Most current \glspl{PPL} already have a \gls{DP} or \gls{CRP} implemented.
sampler: particle algorithm

\subsection{Implementation}
As highlighted in the previous section, recursion is key to denote \gls{BNP} models in a \gls{PPL}.

A function is tail-recursive when the recursive call happens as the final action in a function, in which case it can happen without the function call stack growing. In continuation-passing style, there is no stack – all function calls are tail calls, hence all recursive functions are tail-recursive.

Clojure provides special forms \emph{loop} and \emph{recur} for writing tail-recursive programs. Anglican programs are \gls{CPS}-converted and do not use the stack. Therefore recursive calls in the Anglican \gls{PPL} cannot lead to stack overflow.
Without such a specific \emph{recur} operator, the call stack can exceed its maximum size and yields errors.


\section{generative process construction}
PPL -> be able to sample from prior

\subsection{DP and PY}
\textcolor{red}{Size-biased permutation
All measure
Urn: marginalized out the random -> only partition left
Stick-breaking: hybrid-case, lazely instanciate our observation once they are drawn from it, sized-biased permutation }\\

well known
already implemented in most \gls{PPL}s

For a $\text{PY}(\sigma, \theta)$
\begin{gather*}
Z_1 \sim \text{beta}(1-\sigma,\theta + \sigma) \\
Z_2 \sim \text{beta}(1-\sigma,\theta + 2\sigma) \\
\vdots \\
Z_l \sim \text{beta}(1-\sigma,\theta + l\sigma) \\
\vdots \\
\end{gather*}
the corresponding weights are:
$$ p_l := Z_l \prod_{j<l}{(1 - Z_j)} $$

\subsection{PK}
Theorem 2.1 of Perman et al. states that the sequence of surplus masses $(T_k)_(k>=0)$ forms a Markov chain and gives the corresponding initial distribution and transition kernels.

Could we sample the $K$th stick length in generic way since we know the density ? For instance by restricting to Levy measure intensity and Total mass density which are tractable (not σ-stable PK). One may use a simple rejection sampling with proposal $U(0, t_K)$ (what would be M ?).

There exist other models based directly on specifying a valid distribution for the stick breaks (e.g. Ottowa sequence, Beta-Stacy, others?); these should fit in without a problem.



we reviewed the PKP generative process from Figure \ref{fig:PK_generative_process}, it is reminiscent of the well-known stick breaking construction from Ishwaran and James (2001), where a stick of length one is broken, as in Figure \ref{fig:PY_stick_breaking}, but it is not the same. As mentioned previously, we can reparameterise the model, starting with Equation (2.24), and obtain the corresponding joint distribution in terms of $N$ $(0,1)$-valued stick-breaking weights $\{\pi_j \}_{j=1}^N$, where $N$ is the number of represented sticks after trunctation (Favaro and Walker, 2012). This joint distribution is for a general Lévy measure $\rho$, density $f_\rho$ and it is conditioned on the valued of the random variable $T$. We can then recover the well-known Stick breaking representations for the Dirichlet and Pitman-Yor processes, for a specific choice of $\rho$, if we integrate out $T$.

However, in general, these stick random variables, denoted by $Z_l := \frac{\pi_l}{1 - \sum_{i<l}{\pi_l}}$ form a sequence of dependent random variables with a complicated distribution, except for the two previously mentioned processes, see Pitman (1996) for details.

\begin{gather*}
T \sim \gamma \\
\tilde{J}_1|T \sim \text{SBS}(T) \\
\tilde{J}_2|T,\tilde{J}_1 \sim \text{SBS}(T - \tilde{J}_1) \\
\vdots \\
\tilde{J}_{l}|T,\tilde{J}_1,\dots,\tilde{J}_{l-1} \sim \text{SBS}(T - \sum_{i<l} \tilde{J}_i) \\
\vdots \\
\end{gather*}
the corresponding weights are:
$$ \pi_l := \frac{\tilde{J}_{l}}{T - \sum_{i<l} \tilde{J}_i} $$


$\beta_i = \frac{\tilde{J}_i}{T_{i-1}}$
$\tilde{p_i} = \beta_i \prod_{j=1}^i (1-\beta_i)$


\subsection{PG}
cf article Ben

\section{Models of interest}
calculus for SMC for BNP mixture with fixed parameters
fixed parameters are not a assumption, since can be made random then with PMCMC

write model as streaming/online to avoid IS and take advantage of resampling
as state-space model

PPL's code ?

\section{Sampler}
Now that we have detailed and implemented a generative process (i.e. sampling from the prior) for our class of models, we focus in this section on the sampler scheme to use within the \gls{PPL}.

\textcolor{red}{paper reference for particle methods for high dimensional space}
It has been shown that particle methods are the class of sampler scheme to use in the case of high dimensional space.
We are interested in the more general Bayesian case were the parameters are also random variables and to be inferred. Consequently, \acrlong{PMCMC} are methods of choice.

Yet, there is a big path degeneracy issue.

PG -> path degeneracy
PPMH -> ?
PGAS / IPMCMCM


\section{Open questions}
Some questions are still open on this subject, and we are currently working on these.

Is a stick-breaking-like Markov Chain necessary and sufficient for doing inference with particle methods?


What should be the representation of the state in the PPL (theoretically and efficiently concerned) ? Unique components + assignments | all components | sticks lengths ? 

Other models than infinite mixture models  
