\documentclass[twoside,10pt,openany,a4paper]{rapport}

\input{preamble/preamble}
\newacronym{VI}{vi}{variational inference}
\newacronym{KL}{kl}{Kullback-Leibler}
\newacronym{ELBO}{elbo}{\emph{evidence lower bound}}
\newacronym{MCMC}{mcmc}{Markov chain Monte Carlo}

\newacronym{BNP}{bnp}{Bayesian Nonparametric}
\newacronym{PPL}{ppl}{Probabilistic Programming Language}
\newacronym{PPLs}{ppls}{Probabilistic Programming Languages}
\newacronym{PL}{pl}{Programming Languages}

\newacronym{SIS}{sis}{Sequential Importance Sampling}
\newacronym{SMC}{smc}{Sequential Monte Carlo}
\newacronym{PG}{pg}{Particle Gibbs}
\newacronym{PMMH}{pmmh}{Particle Marginal Metropolis-Hastings}
\newacronym{IPMCMC}{ipmcmc}{Interacting Particle Markov Chain Monte Carlo}
\newacronym{PGAS}{pgas}{Particle Gibbs with Ancestor Sampling}

\newacronym{HMC}{hmc}{Hamiltonian Monte Carlo}
\newacronym{HMCDA}{hmcda}{Hamiltonian Monte Carlo with Dual Averaging}
\newacronym{NUTS}{nuts}{No-U-Turn Sampler}
\newacronym{SGLD}{sgld}{Stochastic Gradient Langevin Dynamics}
\newacronym{SGHMC}{sghmc}{Stochastic Gradient Hamiltonian Monte Carlo}

\newacronym{CPS}{cps}{Continuation-Passing Style}
\newacronym{AD}{ad}{Automatic Differentiation}


\input{preamble/preamble_math}
%\input{preamble/preamble_tikz}

\usepackage{algpseudocode,algorithm,algorithmicx}  
\usepackage{caption}
\usepackage{subcaption}

\graphicspath{{images/}}

\newtheorem{theorem}{Theorem}
\newtheorem{proposition}{Proposition}
\newtheorem{definition}{Definition}

\begin{document}

%Non affiché mais sera inséré dans les propriétés du fichier
\title{Césure}
\author{Émile \textsc{Mathieu}}
\date{\today}

%\mainmatter
\frontmatter

%Page de garde
\begin{titlepage}
      \begin{center}
      \begin{figure}
\centering
\begin{subfigure}{.5\textwidth}
  \centering
  \includegraphics[width=.6\linewidth]{logo_enpc.jpg}
\end{subfigure}%
\begin{subfigure}{.5\textwidth}
  \centering
  \includegraphics[width=.6\linewidth]{logo_oxford.png}
\end{subfigure}
\vspace{1.0cm}
\end{figure}

      \vspace{0.3cm}
      \institute{École des Ponts ParisTech}\\
      \institute{Department of Statistics, University of Oxford}
      
      \vspace{0.7cm}
      2017\\
      Master's Internship Report
      
      \vspace{0.3cm}
      Émile Mathieu\\
      Élève-ingénieur, Third year
      
      \vspace{2cm}
      \Huge{\textbf{Bayesian Nonparametric Inference within Probabilistic Programming Languages}}\\

      \vfill
      \normalsize{}
      Internship carried out at Department of Statistics, University of Oxford \\
      %Stages effectués du 6 Juillet au 30 décembre 2015 puis du 11 Janvier au 08 juillet 2016
      From the 22nd of May, to the 15th of September 2017.
      
      \vspace{0.3cm}
       Company tutor: \textsc{Teh}, Yee Whye\\
       Training supervisor: \textsc{Obozinski}, Guillaume

      \end{center}
\end{titlepage}

\cleardoublepage


\chapter{Acknowledgments}

First of all, I would like to express my indebtedness appreciation to my departmental supervisor Prof. Yee Whye Teh. His belief in me and his advices played a decisive role in making the execution of my work and thus this report. \\

I also express my deepest thanks to Benjamin Bloem-Reddy, who as a postdoctoral, oversaw me during this internship and with whom I have continuously worked. \\

Moreover, my gratitude goes to Guillaume Obozinski, my school training supervisor, whose guidance has continually shaped my career path since I have been at Ecole des Ponts ParisTech.

\chapter{Abstract}

On one side, \gls{BNP} models have gained attraction because of their flexibility. These models, which automatically adapt with the number and complexity of data, avoid having to define \textit{a priori} the number of parameters of the model, such as the number of components for a mixture model.
On the other side, \glspl{PPL} allow practitioners to express probabilistic models in a universal way, and bring generic inference algorithms. These systems avoid designing specific inference schemes, which is error-prone and time consuming.
Specific representations of \gls{BNP} models must be used so as to denote such models in \glspl{PPL}, since \gls{BNP} models live in infinite dimensional space and machines only have finite memory and computational resources.
%Secondly, how should the inference schemes be designed so that by using \gls{PPL}s (and thus gaining in flexibility) performance is not too much affected ?

In this report, we review well-known \gls{BNP} models with a focus on mixture models and discrete random probability measures, but we also give background on the design of \glspl{PPL} and associated inference schemes. We utilize generative construction of \glspl{BNP} and show that more generic \gls{BNP} classes than the \acrlong{DP} can be represented in \glspl{PPL}. We prototype our approach by contributing to an existing \gls{PPL}.

\textbf{Keywords :} Probabilistic Programming, Bayesian Non-parametric, Bayesian Inference, Generative process, Sampling methods.


% \chapter{Résumé}

\tableofcontents
\addcontentsline{toc}{chapter}{Table of contents}
%\listoftables
%\addcontentsline{toc}{chapter}{Liste des tableaux}
% \listoffigures
% \addcontentsline{toc}{chapter}{List of figures}

% \chapter{Glossary}

% \begin{itemize} 
% \item \textbf{BNP}:  \textit{Bayesian Non-Parametric}, explained in Section \ref{BNP_def}. \\
% \item \textbf{PPL}:  \textit{Probabilistic Programming Language}, explained in Section \ref{PPL_def}. \\

% \end{itemize}

\mainmatter


\chapter{Introduction}

For data science practitioners, statistical inference is typically just one step in a more elaborate analysis workflow. The first stage of this work involves data acquisition, pre-processing and cleaning. This is often followed by several iterations of exploratory model design and testing of inference algorithms. Once a sufficiently robust statistical model and a corresponding inference algorithm have been identified, analysis results must be post-processed, visualized, and in some cases integrated into a wider production system.
Probabilistic programming systems \cite{Gordon:2014:PP:2593882.2593900,Goodman:2012uq,Mansinghka:2014ty,wood-aistats-2014,Turing} represent generative models as programs written in a specialized language that provides syntax for the definition and conditioning of random variables. The code for such models is generally concise, modular, and easy to modify or extend. Inference can be performed for any probabilistic program using one or more generic inference techniques provided by the system back end. These systems therefore avoid having to design specific inference schemes, which is error-prone and time consuming.

In the previously described data scientist's workflow, model selection is often a particularly difficult step.
Most scientists address this problem by first fitting several models, with different numbers of clusters or factors, and then selecting one using model comparison metrics \cite{Claeskens:1251912}. Model selection metrics usually include two terms. The first term measures how well the model fits the data. The second term, a complexity penalty, favours simpler models (i.e., ones with fewer components or factors).
\acrlongpl{BNP} models provide a different approach to this problem. Rather than comparing models that vary in complexity, the \gls{BNP} approach is to fit a single model that can adapt its complexity to the data. Indeed, they allow the complexity to grow as more data are observed, such as when using a model to perform prediction. This is an attractive property for many settings.
Nonetheless, such models cannot be used out-of the box within \acrlongpl{PPL} since they have by definition at least one infinite dimensional component and computers only have finite memory and computational resources. Consequently, specific representations of these models must be constructed so as to denote them in probabilistic programming systems. Our work focuses on generative constructions of such models, and on developing efficient algorithms to perform inference within \glspl{PPL}.

After giving some background on the Department of Statistics in Chapter 1 we develop on the context and organisation of the project. We then review in Chapter 4, well-known \acrlong{BNP} models, with a focus on infinite mixture models and discrete random probability measures. This chapter also highlights the tailored representations of \gls{BNP} models used by \acrlong{MCMC} samplers in the literature. After that in Chapter 5, we review the design of \acrlongpl{PPL} and describe the framework in which inference schemes can be generically applied for these programs.
We then recall a generative construction of \acrlongpl{PKP} in Chapter 6, and stress out how it relates with a specific class of \glspl{PPL}. We also implement this generative process, along with a mixture model in an already existing \gls{PPL}, and run experiments to assess the performance of posterior samplers.
Finally, in Chapters 7 and 8 we conclude an point out possible directions for future work.

The main contributions of this report are the following:

\begin{itemize}
  \item Being able to sample from some discrete random probability measures in \glspl{PPL}, and therefore represent infinite mixture models.
  \item Efficiently sample from posterior distributions of hidden variables in infinite mixture models.
\end{itemize}

\chapter{Presentation of the Department of Statistics}


\section{Creation}
\quad The Department of Statistics \footnote{\url{https://www.stats.ox.ac.uk}} is part of the University of Oxford, along with the other departments and the 38 constituent colleges.
The University of Oxford was founded in the 11th century, which makes it the oldest university in the English-speaking world and the world's second-oldest university in continuous operation.

The Department of Statistics was officially created in 1988, even though first moves in the development of Oxford statistics can be dated to the 19th century.

Indeed, In the 1870s, Florence Nightingale -- the pioneer of modern nursing -- discussed the possibility of endowing a Professorship of Statistics in Oxford, but the proposal eventually foundered.
However, Oxford did appoint a statistician to a chair in 1891, although not to a chair in statistics.

The next significant moves in the development of Oxford statistics were by economists, who were increasingly keen to build economic theory on a foundation of sound data analysis.
This led to the creation in 1935 of an Institute of Statistic, swhich was then renamed as the Institute of Economics and Statistics in 1962. 

The sequence of events which led directly to the establishment of the present Department of Statistics began with the appointment in 1945 of David Finney as the university’s first Lecturer in the Design and Analysis of Scientific Experiment (LIDASE).

Then in the 1980s, after the Department of Biomathematics' head increasingly felt that Oxford was losing out in the face of developments in statistics, a working party appointed by the general board of the university to assess a careful analysis of the organisation of statistics in Oxford. 
They found fragmentation to be ‘the dominant feature of Oxford statistics’ and concluded that ‘fragmentation has serious disadvantages ....’.
The working party’s report recommended the creation of a university statistics department, which were to include the former Department of Biomathematics, together with a new Professorship in Statistical Science and the two existing lecturerships in statistics within the Mathematical Institute.

These major recommendations were all accepted by the university and the new Department of Statistics was created in 1988.

\section{Activities}
\quad The Department of Statistics at Oxford is a world leader in research including computational statistics and statistical methodology, applied probability, bioinformatics and mathematical genetics.
The main research groups in the Department are Computational statistics and machine learning, Probability, Statistical genetics and bioinformatics, Protein Informatics and  Statistical Genetics.

I am part of the Computational Statistics and Machine Learning Group (OxCSML) \footnote{\url{http://csml.stats.ox.ac.uk/people/mathieu/}}, which have research interests spanning Statistical Machine Learning, Monte Carlo Methods and Computational Statistics, Statistical Methodology and Applied Statistics. \\

The department offers an undergraduate degree (BA or MMath) in Mathematics and Statistics, jointly with the Mathematical Institute. 
At postgraduate level there is an MSc course in Applied Statistics (MSc in Statistical Science from 2017), as well as a lively and stimulating environment for postgraduate research (DPhil or MSc by Research).
The department also has a consulting activity called \textit{Oxford University Statistical Consulting}.

%\section{Atmosophere} ?
% Department day, brunch, lunches, pub
%Reading Group ?



%!TEX root = internshipReport.tex

\chapter{Mission}
\section{Themes of research}
Prof. Yee Whye Teh \footnote{\url{https://www.stats.ox.ac.uk/~teh}} has worked for a long time on inference sampling schemes for \gls{BNP} mixture models \cite{Favaro:2013fl, Favaro:2014kg, Lomeli:2015vd, Lomeli:2017kp}, but also on stick-breaking constructions \cite{stick-breaking-ibp, Favaro:2014bo}.
He also has recently been interested in \gls{PPL} and consequently in inference schemes within \gls{PPL} for \gls{BNP} models.
This theme requires knowledge in several fields -- Probabilities, Computational Statistics, Programming Languages -- which makes it deeply interesting.
He proposed me working with him on this topic as part of a 3-years DPhil program, and to start earlier as an intern.

\section{Context}
In addition to inviting me to work with him, Prof. Yee Whye Teh also opened two postdoctoral positions for working on the same project, which have been filled by Tom Rainforth and Benjamin Bloem-Reddy. 
Tom Rainforth \footnote{\url{http://www.robots.ox.ac.uk/~twgr}} is finishing is third year of DPhil in the Dept. of Engineering Science in Oxford, supervised by Prof. Frank Wood. His interests include  probabilistic programming, Bayesian optimization, probabilistic numerics, sequential Monte Carlo and particle Markov Chain Monte Carlo methods. He will join the group in October, but he has already attended several reading group meetings.
On the other hand, Benjamin Bloem-Reddy \footnote{\url{http://www.stats.ox.ac.uk/~bloemred/}} arrived in May in Oxford and has already started working on the project. He was supervised by Peter Orbanz at Columbia University, and his research were focused on probabilistic and statistical analysis of networks and other discrete data.

\section{Reading group}

Four reading groups are organised with a bi-weekly period: Kernel methods, Deep Learning, Bayesian Nonparametrics and Probabilistic Inference.
I have been leading the Probabilistic Inference reading group \footnote{\url{https://github.com/BigBayes/oxsml/wiki/Probabilistic-Inference-meetings}} since July. Since, Ben and I have presented four papers \cite{Ritchie:2016ve, Edward, Ritchie:2015tx, DelMoral:2015jk} with an emphasise on probabilist programming.

\section{Organisation}
In this section I develop my current organisation and workflow as a researcher.
At first, I had much trouble to organise my workflow, I wrote my papers' review and new ideas on flying sheets, papers were saved in my computer's folders, citations for report was time-consuming, my code was locally saved, etc... 
Thus, I worked on a better workflow and after trials and errors, I eventually arrived on what I describe below.
I aim to modify this process with time, so as to continually enhance my productivity and be able to focus on the interesting part of the job.

\subsection{Managing papers}
My biggest trouble was keeping organised the dozens of new articles I read each week.
I was saving them in a tree-like structure of folders, but with the number of articles saved growing, it became more and more difficult to find specific article.
Moreover, this structure inherently prohibits cross-categories articles which is annoying for a project situated at the intersections of several fields.
Furthermore, I had no fast way to cite an article, neither in plain format (for markdown \footnote{\url{https://en.wikipedia.org/wiki/Markdown}} files for instance) nor in \emph{BibTeX} format.

Then, I heard of papers managing library such as Papers3 \footnote{\url{https://www.readcube.com/papers/mac/}} or Mendeley \footnote{\url{https://www.mendeley.com}}.
I have eventually opted for Papers3 but Mendeley is also a popular choice in the academic community.
These applications features many tools easing the life of a researcher, the main one being from my point of view:

\begin{itemize}
\item Synchronisation: between multiple computers or devices.
\item Multi-labels: these are used in the search tool.
\item Local search: search in titles, authors, labels and even papers' content.
\item Online search: can import articles in a fast manner by being connected with online search engines such as \emph{arXiv}.
\item Collections: create a reading list, or group of papers which can be cited at once
\item Citations: get \emph{BibTeX} reference or \emph{BibTeX} cite command in clipboard
\end{itemize}

\subsection{Managing research}
Another of my organizational issue was keeping track of ideas.
I happened to find that research is a result of a long chain of ideas which were continually iterated upon.
I am now maintaining a single \textit{master document} for keeping tracks of this chain of ideas.

It has a bulleted list of all ideas, problems, and topics that I’d like to think more carefully about. This list is succinct but subsequent sections go in depth on particular entries.
This list is sorted according to what I’d like to work on next, but I continually revise my priorities according to whether I think the direction aligns with my broader research vision, and if I think the direction is necessarily impactful for the community at large.

\subsection{Managing projects}
Then, when an idea has matured enough and I have seriously started working on it, I create a Github \footnote{\url{https://github.com}} repository for the project. Each project has its separated repository. 
It contains a \texttt{/readme.md} file maintaining a list of todo’s, with also questions (and sometimes answers!) both for myself and collaborators. This makes it transparent how to keep moving forward and what’s blocking the work.
There is also a \texttt{/doc/} folder for all the write-ups, usually in \LaTeX  format.
The  \texttt{etc/} folder is used for everything not relevant to other directories such as pictures of whiteboards during conversations about the project.
Finally, the \texttt{/src/} folder is where all code is written. Runnable scripts are written directly in \texttt{/src/} , and classes and utilities are written in \texttt{/src/codebase/}.

\subsection{Writing scientific documents}
Concerning the writing of scientific documents, \LaTeX is the language of choice. Yet the commonly used editor for OSX -- TeXShop -- is missing some useful features.
Hopefully, LaTeXTools \footnote{\url{https://github.com/SublimeText/LaTeXTools}}, a LaTeX plugin for Sublime Text \footnote{\url{https://www.sublimetext.com}} has been developed. It has the following useful features; (i) a build command to compile LaTeX, (ii) forward and inverse search with PDF previewers, (iii) fill everything including references, citations, packages, graphics, figures, etc, (iv) smart command completion for a variety of text and math commands, (v) full support for project files and multi-file documents.
I have gained in productivity since I started using Sublime with this plugin instead of TeXShop.



\section{Bayesian nonparametric}

\subsection{Definition} \label{BNP_def}

\subsection{Usefulness}

\subsection{Canonical models}
???
DP, PYP, etc

\subsection{MCMC Inference}
% Variational inference ? DPM \cite{DPVI}

\quad Constructing MCMC schemes for models with one or more Bayesian nonparametric components is an active research area since dealing with the infinite dimensional component $P$ forbids the direct use of standard simulation-based methods.These methods usually require a finite-dimensional representation. The general idea for designing inference schemes is to find finite dimensional representations to be able to store the model in a computer with finite capacity.

There are two main sampling approaches to facilitate simulation in the case of Bayesian nonparametric models: random truncation and marginalisation. These two schemes are known in the literature as conditional and marginal samplers.

\subsubsection{Marginal Samplers}
\quad Marginal samplers bypass the need to represent the infinite-dimensional component by marginalising it out. These schemes have lower storage requirements than conditional samplers because they only store the induced partition, but could potentially have worse mixing properties.

\subsubsection{Conditional Samplers}
\quad Conditional samplers replace the infinite-dimensional prior by a finite-dimensional representation chosen according to a truncation level. Since these samplers do not integrate out the infinite-dimensional component, their output provides a more comprehensive representation of the random probability measure.
thinning vs stick-breaking

\subsubsection{Hybrid Samplers}
YW paper on PK ?

\subsubsection{SMC}
Review of SMC ?
cf Maria Lomeli thesis


\subsection{BNP sampling in PPL}

Stochastic Memoization with DPmem: $\alpha = 0$, deterministic memoization, $\alpha = \inf$ no memoization\\
https://probmods.org/chapters/12-non-parametric-models.html


\subsection{Link between BNP and High order PPL}
See Frank Wood meeting


% \include{sectionInference}

\chapter{Probabilistic programming}

\section{What is it ?} \label{PPL_def}
At a high level, \gls{PPL} are \gls{PL} techniques to abstract inference algorithms from statistics such that they apply automatically and correctly to the broadest possible set of model-based reasoning applications.

A bit more precisely, Probabilistic programming systems \cite{Goodman:2012uq,dippl,Mansinghka:2014ty,wood-aistats-2014} represent generative models as programs written in a specialized language that provides syntax for the definition and conditioning of random variables.

Indeed, ``probabilistic programs are usual functional or imperative programs with two added constructs: 
(1) the ability to draw values at random from distributions, and 
(2) the ability to condition values of variables in a program via observations.'' \cite{Gordon:2014:PP:2593882.2593900}

\textcolor{red}{Schema of inference / CS / Stats ?}

\section{Why is it useful ?}
The main goal of \gls{PPL}s is to increase productivity. One of the savings is to be found in the amount of code that needs to written in order to prototype and develop models.

Moreover, \gls{PPL}s remove the burden of having to develop inference code for each new model which is famously error-prone and time consuming.
This is done by providing a modeling language abstraction layer in which developers can denote their models.  Once denoted, generic inference is provided for free.

\section{Existing languages}
The first generation of \gls{PPL}s had limitation in the range of models that could be represented and in which inference could be performed.
BUGS \cite{Bugs} and STAN \cite{Stan} can only work with graphical models.
Similarly Factorie \cite{Factorie} and Infer.NET \cite{InferNET} only handle factor graphs.
These so-called \textit{First-Order} \gls{PPLs}, can only represent finite dimensional model and have bounded loops.
\textcolor{red}{Schema of first order and high order PPLs ?}

On the other hand, \textit{High Order \gls{PPLs}} which arrived a bit before the 2010s, are Turing complete, allow complex control flow, including stochastic recursion, thus can denote infinite dimensional objects.
They are easy to program in, natural to express certain models, but it is hard to perform inference in
these \gls{PPL}s. Anglican \cite{wood-aistats-2014}, Venture \cite{Mansinghka:2014ty}, Church \cite{Goodman:2012uq} and WebPPL \cite{dippl} are \textit{High Order \gls{PPLs}}.


Recently, a new \gls{PPL} named Edward \cite{Edward} has been developed. It is different from the classical \gls{PPL}s since it focuses on \gls{VI} and Hamiltonian methods.

\section{Design}
Probabilistic programs denote probabilistic generative models as programs that include \texttt{sample} and \texttt{observe} statements. Both \texttt{sample} and \texttt{observe} are functions that specify random variables in this generative model using probability distribution objects as an argument, while \texttt{observe}, in addition, specifies the conditioning of this random variable upon a particular observed value in a second argument. These observed values induce a conditional probability distribution over the execution traces whose approximations and expected values we want to characterize by \textit{performing inference}.

An execution trace of a probabilistic program is obtained by successively executing the program deterministically, except when encountering \texttt{sample} statements at which point a value is generated according to the specified probability distribution and appended to the execution trace.
We denote the observed values by $\mby := (y_j)_{j=1}^N$.

Depending on the probabilistic program and the values generated at sample statements, the order in which the execution encounters sample statements as well as the number of encountered sample statements may be different from one trace to another.

A (almost-surely terminating) probabilistic program defines a probability distribution over finite feasible traces x with probability density $\pi(\mbx) := \gamma(\mbx) / Z$ where 

\begin{equation*}
\gamma(\mbx) := \prod_{i=1}^{|\mbx|}{f(x_i~|~x_{1:i-1})} \prod_{j=1}^{|\mby|}{g(y_j~|~x_{1:\tau(j)})}
\end{equation*}

and Z is the normalizing constant $Z := \int{\gamma(\mbx) d(\mbx)}$.

\section{Inference}
\textcolor{red}{Add pseudo-code for each algo ?}

Inference in probabilistic programming characterizes the conditional distribution of such variables given observed data assumed to have been generated by executing the probabilistic program.

Typically inference can be performed for any probabilistic program using one or more generic inference techniques provided by the system back end, such as Metropolis-Hastings \cite{Wingate:2011ul, Mansinghka:2014ty}, Hamiltonian Monte Carlo \cite{Stan}, expectation propagation \cite{InferNET}, and extensions of Sequential Monte Carlo \cite{vandeMeent:2015uk, Paige:2014tc, Wood:2015tr} methods.

\subsection{Use-case}
Even if as highlighted before, inference in \gls{PPL}s should be able to deal with arbitrary series of targets, for simplicity we will focus on a non-Markovian \gls{SSM}.

\Gls{SSM}s are probabilistic models over a set of latent variables $X_t \in \mathcal{X}_t, \forall t = 1 : T$ and observed variables $Y_t \in \mathcal{Y}_t , \forall t = 1 : T$ . We can further consider a model to be parameterized by $\theta \in \Theta$. The \gls{SSM} is then characterized by an initial density $\mu_\theta(x_1)$, a series of transition densities $f_{t,\theta}(x_t|x_{1:t-1})$, and a series of emission densities $g_{t,\theta}(y_t|x_{1:t})$.

\begin{equation*}
\begin{aligned}
& X_1 \sim \mu_\theta(\cdot) \\
& X_t|(X_{1:t-1} = x_{1:t-1}) \sim f_{t,\theta}(\cdot|x_{1:t-1}) \\
& Y_t|(X_{1:t} = x_{1:t}) \sim g_{t,\theta}(\cdot|x_{1:t}) \\
\end{aligned}
\end{equation*}

The joint density of the \gls{SSM} is then as follows

$$ p_\theta(x_{1:T},y_{1:T}) = \mu_\theta(x_1) \prod_{t=2}^T f_{t,\theta}(x_t|x_{1:t-1}) \prod_{t=1}^T g_{t,\theta}(y_t|x_{1:t}) $$

We are free to choose any density for $\mu_\theta(x_1)$ and each $f_{t,\theta}(x_t|x_{1:t-1})$ and $g_{t,\theta}(y_t|x_{1:t})$. One is usually interested characterizing the posterior

$$ p_\theta(x_{1:T}|y_{1:T}) \propto p_\theta(x_{1:T},y_{1:T}) $$

Or expectations of some function $\phi$ under this posterior

$$ I(\phi) = \int \phi(x_{1:T}) p_\theta(x_{1:T}|y_{1:T}) dx_{1:T} $$


\subsection{Importance Sampling}
\textcolor{red}{Detail SMC and PMCMC}

\paragraph{\gls{IS}}
Importance sampling is an example of a Monte Carlo sampling scheme that provides approximately independent and identically distributed samples from a distribution of interest or target distribution, such as a posterior distribution, by generating a candidate sample from a proposal or importance distribution $q(\mbx|y_1,\dots,y_T)$.
The fact that the weights 
$w^k = \frac{p(\mbx, y_1,\dots,y_T)}{q(\mbx|y_1,\dots,y_T)} \propto \frac{p(\mbx|y_1,\dots,y_T)}{q(\mbx|y_1,\dots,y_T)}, \ \text{with} \ k \in 1,\dots,K$ can be computed is exploited, and samples from the target are obtained by sampling from the following weighted empirical distribution

$$ \hat{p}(\mbx|y_1,\dots,y_T) = \sum_{k=1}^K \bar{w}^k \delta_{\tilde{\mbx}^k} (\mbx)$$

where $\bar{w}^k = \frac{{w}^k}{\sum_{k=1}^K {w}^k}$ is the normalized weight and
$\delta_{z}$ is a Dirac measure centered on $z$.
The expectation $I(\phi)$ can also be approximated using

$$ \hat{I}(\phi) = \sum_{k=1}^K \bar{w}^k \phi(\tilde{\mbx})$$

% \begin{algorithm}  
%   \caption{ImportanceSampling(f, g)
%     \label{alg:is}}  
%   \begin{algorithmic}[1]  
%     \State  $\mu \gets 0_d$
%       \For{$t \gets 1 \textrm{ to } T$}  
%           \State  $\beta_t \gets 2 \log (\left| D \right| t^2 \pi^2 / 6 \delta)$
%     \State Choose $a_t \gets arg \max_i \mu_{t-1} + \sqrt{\beta_t} \sigma_{t-1}$
%     \State Observe $y_t = f(\mathbf{x}_t) + \epsilon_t$
%     \State $\mu_t = k_{t-1}(\mathbf{x})^T {(K_{t-1} + \sigma^2 I_d)}^{-1} y_t$
%     \State $k_t = k(\mathbf{x}, \mathbf{x}') - k_{t-1}(\mathbf{x})^T {(K_{t-1} + \sigma^2 I_d)}^{-1} k_{t-1}(\mathbf{x'}) $
%     \State $\sigma_{t}^2 = k_t(\mathbf{x},\mathbf{x})$
%       \EndFor  
%   \end{algorithmic}  
% \end{algorithm}

One problem with this method is that it is not easy to choose the proposal distribution $q$. A good proposal should share most of the support of the target distribution and have the same number of modes, i.e. it should be “close” to the target. A second problem is that it is a batch estimation method. To tackle this latter issue, in the next section, an extension to a sequential scenario is described.

\paragraph{\gls{SIS}}
\gls{SIS} exploits the structure of a model by breaking down the overall inference problem into a series of target distributions which get incrementally closer to the distribution of interest.
These targets are then approximated by propagating a population of samples known as particles. If each intermediary target is kept similar to its predecessor, approximating one target given the last forms a significantly simpler problem than the overall inference.

More formally, \gls{SIS} performs approximate inference on a sequence of target distributions
$\left(\pi_t(x_{1:t}) \right)_{t=1}^T$ of increasing spaces 
$\left(\mathcal{X}_1 \times \dots \times \mathcal{X}_t \right)_{t=1}^T$.
In the context of \gls{SSM}s, the target distributions are often taken to be
$\left(p_\theta(x_{1:t}|y_{1:t}) \right)_{t=1}^T$.
At each time step $t$, we have a set of $K$ particles $\left(\tilde{x}_{1:t}^k \right)_{t=1}^T$,
corresponding to samples of the latents, and respective particle weights $\left({w}_{t}^k \right)_{t=1}^T$.
Similarly to \gls{IS}, using these weighted particles, one can approximate each posterior
$p_\theta(x_{1:t}|y_{1:t})$.
In particular, the posterior for the complete model $p_\theta(x_{1:T}|y_{1:T})$ and the expectation $I(\phi)$ can be approximated using the following estimators

$$ \hat{p}(x_{1:T}|y_{1:T}) = \sum_{k=1}^K \bar{w}_T^k \delta_{\tilde{x}_{1:T}^k} ({x}_{1:T})$$

$$ \hat{I}(\phi) = \sum_{k=1}^K \bar{w}_T^k \phi(\tilde{x}_{1:T})$$

where $\bar{w}_T^k = \frac{{w}_T^k}{\sum_{k=1}^K {w}_T^k}$ is the normalized weight.



Let us assume that the importance distribution $q$ at time $t$ depends on all data points until time $t$ and not on the future data points, the joint posterior can be written in the following factorised form

$$ q(x_{1:T}|y_{1:T}) = q(x_1) \prod_{t=2}^T q(x_t|x_{1:t-1},y_{1:t}) $$

The corresponding importance weight is

\begin{equation} \label{eq:IS_w}
\begin{aligned}
w_t =& \frac{p(x_{1:T}|y_{1:T})}{q(x_{1:T}|y_{1:T})} \\
w_t =& \frac{\mu(x_1) \prod_{t=2}^T p(x_t|x_{1:t-1},y_{1:t})}{q(x_1) \prod_{t=2}^T q(x_t|x_{1:t-1},y_{1:t})} \\
\end{aligned}
\end{equation}

The target distribution is the posterior distribution up to time $t$, which changes sequentially as we observe more data.


The posterior distribution can be estimated recursively due to

\begin{equation} \label{eq:IS_p_rec}
p(x_{1:t+1}|y_{1:t+1}) = p(x_{1:t}|y_{1:t}) \times \frac{p(y_{t+1}|x_{1:t+1})p(x_{t+1}|x_{1:t})}{p(y_{t+1}|y_{1:t})} \\
\end{equation}

Substituting the numerator of Equation \ref{eq:IS_p_rec} in \ref{eq:IS_w} we obtain a recursive equation for the importance weight at time $t+1$

$$ w_{t+1} = w_{t} \times \frac{p(y_{t+1}|x_{1:t+1})p(x_{t+1}|x_{1:t})}{p(y_{t+1}|y_{1:t})} $$


Even if we are not dealing with a \gls{SSM}, \gls{SIS} can be used as a general-purpose algorithm. The data are assumed to be observed sequentially so the observation’s index is the time index.

\textcolor{red}{PPL breakpoints}
\textit{breakpoints} are needed. \gls{CPS} in \emph{Anglican} \cite{wood-aistats-2014} and \emph{WebPPL} \cite{dippl}
coroutine copying in Turing \cite{Turing}

\paragraph{\gls{SMC}}
\textcolor{red}{Illustration of particles wrt time t ?}

The main problem with \gls{SIS} is that the weights become more skewed as the number of data points increases \cite{Anonymous:2001ue}, after a few steps only one particle will have a significant weight. To remedy this, a resampling step can be introduced which allows to eliminate particles with small weights and replicate particles with high weights. This selection step can be introduced only occasionally or at every step of the algorithm. We resample from the empirical posterior at step $t$

$$ \hat{p}(x_{1:T}|y_{1:T}) = \sum_{k=1}^K \bar{w}_t^k \delta_{\tilde{x}_{1:T}^k} ({x}_{1:T})$$

If the selection step is to be performed occasionally, a possible criterion is when the \gls{ESS} is below a given threshold, which is a function of the number of particles, a popular choice is $0.5T$. The \gls{ESS} for the unnormalised weights is given by

$$ ESS_t = \frac{{\left( \sum_{k=1}^K{w_i^k} \right)}^2}{\sum_{k=1}^K{{w_i^k}^2}}$$




The resampling step is achieved by, at each time step $t = 2, \dots , T$ ,
selecting the ancestor index $a^k_{t-1}$ for the $k$th particle from a discrete distribution 
$\mathcal{F} (\cdot|\bar{w}^1_{t-1},\dots, \bar{w}^K_{t-1})$
over parent indices ${1, \dots , K }$ with probabilities equal to the normalized weights at the previous time step $(\bar{w}^1_{t-1},\dots, \bar{w}^K_{t-1})$. \cite{Douc:2005wa} provides a comparison of numerous different schemes
for sampling from $\mathcal{F} (\cdot|\bar{w}^1_{t-1},\dots, \bar{w}^K_{t-1})$ that reduce the variance of the \gls{SMC} estimates compared with naïve multinomial resampling.


\subsection{\gls{PMCMC}}
In a Bayesian setting, it is usual to consider the parameter $\theta$ in $p_\theta(x_{1:T}|y_{1:T})$ as a random variable by specifying a prior $p(\theta)$. In this subsection, we are therefore interested on algorithms targeting
$p(\theta, x_{1:T}|y_{1:T})$.
Let's think about \gls{MCMC} algorithms targeting the distribution $p(\theta, x_{1:T}|y_{1:T})$ which rely on sampling exactly from $p_\theta(x_{1:T}|y_{1:T})$, called ‘idealized’ algorithms.
Such algorithms are purely conceptual but a natural idea consists of approximating these idealized algorithms by using the output of an \gls{SMC} algorithm targeting $p_\theta(x_{1:T}|y_{1:T})$ using $K \ge 1$ particles as a proposal distribution for an \gls{MH} update.
Intuitively this could allow us to approximate with arbitrary precision such idealized algorithms while only requiring the design of low dimensional proposals for the \gls{SMC} algorithm.

A direct implementation of this idea is impossible as the marginal density of a particle that is generated by an \gls{SMC} algorithm is not available analytically but would be required for the calculation of the \gls{MH} acceptance ratio. Yet the \gls{SMC} algorithm yields an unbiased estimate of the marginal likelihood

$$ \hat{p}(y_{1:T}) = \prod_{t=1}^T \frac{1}{K} \sum_{k=1}^K w_t^k $$

which can be used for the calculation of the \gls{MH} acceptance ratio. 
These \gls{PMCMC} updates have been introduced in \cite{Andrieu:2010gc}.
The key feature of PMCMC algorithms is that they are in fact ‘exact approximations’ to idealized \gls{MCMC} algorithms targeting either $p(\theta, x_{1:T}|y_{1:T})$ in the sense that for any fixed number $N\ge1$ of particles their transition kernels leave the target density of interest invariant. 


\paragraph{\gls{PMMH}}
\gls{PMMH} makes use of the standard decomposition $p(\theta, x_{1:T}|y_{1:T}) = p(\theta | y_{1:T}) p_\theta(x_{1:T}|y_{1:T})$. 
It is natural to suggest the following form of proposal density for an \gls{MH} update:

$$ q\left( \theta^\star, x_{1:T}^\star | \theta, x_{1:T} \right) = q(\theta^\star|\theta) p_{\theta^\star}(x_{1:T}^\star|y_{1:T})$$

for which the proposed $x_{1:T}^\star$ is given by a \gls{SMC} algorithm targeting $p_{\theta^\star}(x_{1:T}|y_{1:T})$. Thus, the only degree of freedom of the algorithm (which will affect its performance) is $q(\theta^\star|\theta)$. The resulting \gls{MH} acceptance ratio is given by

$$ 1 \wedge \frac{p_{\theta^\star}(y_{1:T}) p(\theta^\star) q(\theta|\theta^\star)}{p_{\theta}(y_{1:T}) p(\theta) q(\theta^\star|\theta)} $$

\gls{PMMH} uses $\hat{p}(y_{1:T})$ to compute the acceptance ratio. It has been proven \cite{Andrieu:2010gc} that the PMMH update leaves $p(\theta, x_{1:T}|y_{1:T})$ invariant and that under weak assumptions the \gls{PMMH} sampler is ergodic.


\paragraph{\gls{PG}}
An alternative to the previous algorithm to sample from $p(\theta, x_{1:T}|y_{1:T})$ consists of using the Gibbs sampler which samples iteratively from $p(\theta|x_{1:T},y_{1:T})$ and $p_\theta(x_{1:T}|y_{1:T})$.
It is often possible to sample from $p(\theta|x_{1:T},y_{1:T})$ and thus the potentially tedious design of a proposal density for $\theta$ that is necessary in the \gls{PMMH} update can be bypassed.

It has been shown \cite{Andrieu:2010gc} that the naïve particle approximation to the Gibbs sampler where sampling from $p_\theta(x_{1:T}|y_{1:T})$ is replaced by sampling from an SMC approximation $\hat{p}_\theta(x_{1:T}|y_{1:T})$ does not admit $p(\theta, x_{1:T}|y_{1:T})$ as invariant density.

A valid particle approximation to the Gibbs sampler requires the use of a special type of \gls{PMCMC} update called the \textit{conditional} \gls{SMC} update. This update is similar to a standard \gls{SMC} algorithm but is such that a prespecified path $x^\star_{1:T}$ is ensured to survive all the resampling steps, whereas the remaining $N-1$ particles are generated as usual.

\paragraph{\gls{PGAS}}
A drawback of \gls{PG} is that it can be particularly adversely affected by path degeneracy in the \gls{CSMC} step. Conditioning on an existing trajectory means that whenever resampling of the trajectories results in a common ancestor, this ancestor must correspond to this trajectory. Consequently, the mixing of the Markov chain for the early steps in the state sequence can become very slow when the particle set typically coalesces to a single ancestor during the \gls{CSMC} sweep.

\cite{Lindsten:2014uw} introduces \gls{PGAS}, which alleviates the problem with path degeneracy by modifying the original \gls{PG} kernel with a so-called \gls{AS} step.
The idea is to sample a new value for the index variable $a_t^N$ in an ancestor sampling step. While this is a small modification of the algorithm, the improvement in mixing can be quite considerable.
The task is to artificially assign a history to the partial path $(x^\star_{t:T})$ of the reference path.
This is done by connecting $(x^\star_{t:T})$ to one of the particles $(x^k_{1:t-1})$.

$$ \tilde{w}^k_{t-1|T} = \frac{}{ }  $$

\paragraph{\gls{IPMCMC}}
\gls{IPMCMC} \cite{Rainforth:2016wq} is another way of tackling the path degeneracy issue.
In \gls{IPMCMC}, a pool of \gls{CSMC} and unconditional \gls{SMC} algorithms are ran as parallel processes (referred as nodes. After each run of this pool, successive Gibbs updates are applied to the indexes of the \gls{CSMC} nodes, such that the indices of the \gls{CSMC} nodes changes. Hence, the nodes from which retained particles are sampled can change from one \gls{MCMC} iteration to the next. This lets us trade off exploration (\gls{SMC}) and exploitation (\gls{CSMC}) to achieve improved mixing of the Markov chains.

\subsection{\gls{MCMC}}
specific to \gls{PPL}s
\cite{Wingate:2011ul}
\cite{Ritchie:2015tx}
%Venture (graph MCMC) ?

\subsection{Hamiltonian}
\textcolor{red}{Can be quite brief for Hamiltonian}

\paragraph{\gls{HMC}}


\paragraph{\gls{HMCDA}}
\textcolor{red}{useless ? because of NUTS}

\paragraph{\gls{NUTS}}
In \cite{NUTS}, the authors address the issue of choosing the two hyperparameters of \gls{HMC}: a step size $\epsilon$ and a desired number of steps $L$, since \gls{HMC}'s performance is highly sensitive on those.
\cite{Nesterov2009}

\paragraph{\gls{SGLD}}
mini-batch / online setting, scale to bug dataset
\footnote{See for instance, \gls{SGLD} applied to a Bayesian logistic regression at \url{https://github.com/yebai/Turing.jl/blob/master/example-models/sgld-paper/lr_sgld.jl}}
 \cite{SGLD}

\paragraph{\gls{SGHMC}}
Same setting as \gls{SGLD}
Naive version is wrong (posterior is not the invariant distribution), see \cite{SGMCMC}
friction term
 \cite{SGHMC}
 
\subsection{Variational Inference}
\textcolor{red}{Introduction to VI ? or in appendix ?}

MCMC methods can be slow to converge and their convergence can be difficult to diagnose.
To my knowledge, \emph{Edward} \cite{Edward} is the only \gls{PPL} handling variational inference.


\section{Contributions}
During this internship I have taken the time to actually implement several inference algorithms, and by so, I contributed to two existing \gls{PPL}s.
Some only for the sake of learning more about sampling schemes and \gls{PPL}s, but others as specifically part of the project. 

First, I implemented \footnote{See \url{https://github.com/yebai/Turing.jl/tree/master/src/samplers}} both the \gls{SGLD} and \gls{SGHMC} inference algorithms in Turing.jl \cite{Turing}, a \gls{PPL} based on Julia \cite{Bezanson:2017gd} and developed at the University of Cambridge.
Then, I implemented \footnote{See \url{https://github.com/blei-lab/edward/pull/728}} the Dual Averaging extension \cite{NUTS} of \gls{HMC} for Edward \cite{Edward}, a \gls{PPL} built on top of Tensorflow \cite{Tensorflow} by Blei's group \footnote{\url{http://www.cs.columbia.edu/~blei/}} at Columbia University.

More recently, I have worked on \gls{PMCMC} methods for Turing. \Gls{PMMH} \footnote{\url{https://github.com/yebai/Turing.jl/pull/339}} is implemented but not merged yet, and I am currently working on \gls{PGAS} and \gls{IPMCMC}. Consequently I became a \textit{Collaborator} of the Turing's repository.

I have also written a stick-breaking representation of the Dirichlet Process which inherits the Distribution.jl \footnote{\url{https://github.com/JuliaStats/Distributions.jl}} type so as to be easily used in Turing.

\textcolor{red}{To develop ?}
 





\chapter{\gls{BNP} inference within \glspl{PPL}} \label{BNP_PPL}
In this chapter we focus on the task of performing inference within a \gls{PPL} for models with an infinite dimensional component, also called \gls{BNP} models.

For now we have restricted ourselves to infinite mixture models, but we hope that our framework will enable efficient sampling for other \gls{BNP} models such as the infinite \acrlong{HMM}. We have also focused our study on sampling methods, but \acrlong{VI} is considered in Section \ref{BNP_VI}.

\textcolor{red}{Stochastic Memoization perspective ??}
An interesting perspective of Discrete Random Probability Measure is through \textit{stochastic memoization} \footnote{\url{https://probmods.org/chapters/12-non-parametric-models.html}}.
First let's recall what is the usual (deterministic) memoization.
For instance with the \gls{DP}, with $\alpha = 0$ we recover a deterministic memoization whereas with $\alpha \rightarrow \inf$ there is no memoization at all.


\section{Link between BNP and High order PPL}

In \glspl{PPL}, to transform a variable in a random variable, one only needed to write \texttt{x = sample(Dist(parameters))}. Then the posterior distribution (given some observations) of \texttt{x} will automatically be performed during the inference scheme.
The Bayesian setting thus naturally fits the framework of \glspl{PPL}.
Yet, we believe that there is more  than just a connection between Bayesian inference and \glspl{PPL}, but that there is also a connection between \acrlong{BNP} and High-order \gls{PPL}.

Working with \gls{BNP} models can be tedious because of the constraints of computers.
A \gls{NRM} $\mu$ can be written \cite{Kingman:1967kn} as
$$P = \sum_{k \ge 1}{\tilde{p}_k \delta_{\phi_k}} $$
where $\left(\tilde{p}_k, \phi_k \right)_{k \ge 1}$ is an infinite collection of weights and atoms.
Discrete probability measures with countable support such as \acrlongpl{NRM} cannot be represented in a computer in a naïve manner since a machine has finite memory.
Other representations of such objects must be used, if one hope to denote them in a program.

One key notion in programming languages which will be crucial is the concept of \textit{lazy evaluation} (or \textit{laziness}). It is an evaluation strategy which delays the evaluation of an expression until its value is needed and which also avoids repeated evaluations.
Lazy evaluation is often combined with \textit{memoization}, as described in \cite{Bentley:1982:WEP:539147}. After a function's value is computed for that parameter or set of parameters, the result is stored in a lookup table that is indexed by the values of those parameters; the next time the function is called, the table is consulted to determine whether the result for that combination of parameter values is already available. If so, the stored result is simply returned. If not, the function is evaluated and another entry is added to the lookup table for reuse.

Instead of hoping to construct the entire infinite sequence $\left(\tilde{p}_k, \phi_k \right)_{k \ge 1}$, we will only ``lazily'' sample $\left(\tilde{p}_k, \phi_k \right)$, i.e. when we need them.
Since we work with homogeneous \glspl{CRM}, the $\{\phi_k\}_{k \ge 1}$ are independent and identically distributed according to the base distribution. Sampling the $(\phi_k)_{k \ge 1}$ as we need them is therefore trivial. Yet how could we lazily sample the $(\tilde{p}_k)_{k \ge 1}$ ?
In Subsection \ref{DP}, we presented the stick-breaking process, a generative process for constructing of $(\tilde{p}_k)_{k \ge 1}$.


We stated in Section \ref{PPL_history}, that High-order \glspl{PPL} allow complex control flow, including stochastic recursion. For programming languages, allowing recursion means allowing a function to call itself within the program text. Stochastic or unbounded recursion means that the number of recursive calls is random or unbounded.





Frank's Theorem: "The Lambda abstraction is necessary in a PPL to be able to denote BNP in this language." The Lambda abstraction seems to be synonymous to function recursion and thus allowing stochastic loop, which are useful for BNP.

Stick-breaking process = control flow + recursion = stochastic recursion ?

Talk lambda abstraction
discrete probability with countable support -> cannot represent it in memory
-> lazy recursive representation the measure
-> Urn: posterior predictive distribution of $X_{k+1}$ given $X_0, \dots, X_{k}$
CRP: simple sufficient statistics, but in general not nice like that
All marginal samplers: Urn distributions, involve complicated integral (can sometimes introduce auxiliary variables)

Stick-breaking process: construction to sample a size-biased permutation of the random probability measure

For lazy instantiation (possibly equivalent to denotable / lambda abstraction?)?

want to sample from a discrete distribution with countable support
->finite memory
to represent it, needs a recursive representation

\subsection{Existing work}
Most current \glspl{PPL} already have a \gls{DP} or \gls{CRP} implemented.
sampler: particle algorithm

\subsection{Implementation (TCO)}
As highlighted in the previous section, recursion is key to denote \gls{BNP} models in a \gls{PPL}.

A function is tail-recursive when the recursive call happens as the final action in a function, in which case it can happen without the function call stack growing. In continuation-passing style, there is no stack – all function calls are tail calls, hence all recursive functions are tail-recursive.

Clojure provides special forms \emph{loop} and \emph{recur} for writing tail-recursive programs. Anglican programs are \gls{CPS}-converted and do not use the stack. Therefore recursive calls in the Anglican \gls{PPL} cannot lead to stack overflow.
Without such a specific \emph{recur} operator, the call stack can exceed its maximum size and yields errors.


\section{generative process construction}
PPL -> be able to sample from prior

\subsection{\gls{DP} and \gls{PY}}
\textcolor{red}{Size-biased permutation
All measure
Urn: marginalized out the random -> only partition left
Stick-breaking: hybrid-case, lazely instanciate our observation once they are drawn from it, sized-biased permutation }\\

\textcolor{red}{code of DP stick-breaking ?} 
\begin{lstlisting}[caption={\acrlong{DP} stick-breaking representation},captionpos=b,label=code:DP_SB]
function pickStick(sticks, J) = begin
  return rand(Bernouilli(sticks(J))) ? J : pickStick(sticks, J+1);
end

function makeSticks(alpha) = begin
  var sticks = @memoize((index) ->  rand(Beta(1, alpha)))
  return () -> pickStick(sticks, 1)
end

function DPmem(alpha, baseDist) = begin
  augmentedProc = @memoize((stickIndex) -> rand(baseDist))
  DP = makeSticks(alpha)
  return () -> augmentedProc(DP())
end
\end{lstlisting}

well known
already implemented in most \gls{PPL}s

For a $\text{PY}(\sigma, \theta)$
\begin{gather*}
Z_1 \sim \text{beta}(1-\sigma,\theta + \sigma) \\
Z_2 \sim \text{beta}(1-\sigma,\theta + 2\sigma) \\
\vdots \\
Z_l \sim \text{beta}(1-\sigma,\theta + l\sigma) \\
\vdots \\
\end{gather*}
the corresponding weights are:
$$ p_l := Z_l \prod_{j<l}{(1 - Z_j)} $$

\subsection{PK}
Theorem \ref{prop:perman} (Page \pageref{prop:perman}) from \cite{Perman:1992ke} states that the sequence of surplus masses $(T_k)_{k \ge0}$ forms a Markov chain and gives the corresponding initial distribution and transition kernels

Could we sample the $K$th stick length in generic way since we know the density ? For instance by restricting to Levy measure intensity and Total mass density which are tractable (not σ-stable PK). One may use a simple rejection sampling with proposal $U(0, t_K)$ (what would be M ?).

There exist other models based directly on specifying a valid distribution for the stick breaks (e.g. Ottowa sequence, Beta-Stacy, others?); these should fit in without a problem.



we reviewed the PKP generative process from Figure \ref{fig:PK_generative_process}, it is reminiscent of the well-known stick breaking construction from \ref{sethuraman94}, where a stick of length one is broken, as in Figure \ref{fig:PY_stick_breaking}, but it is not the same. As mentioned previously, we can reparameterise the model, starting with Equation (2.24), and obtain the corresponding joint distribution in terms of $N$ $(0,1)$-valued stick-breaking weights $\{\pi_j \}_{j=1}^N$, where $N$ is the number of represented sticks after trunctation (Favaro and Walker, 2012). This joint distribution is for a general Lévy measure $\rho$, density $f_\rho$ and it is conditioned on the valued of the random variable $T$. We can then recover the well-known Stick breaking representations for the Dirichlet and Pitman-Yor processes, for a specific choice of $\rho$, if we integrate out $T$.

However, in general, these stick random variables, denoted by $Z_l := \frac{\pi_l}{1 - \sum_{i<l}{\pi_l}}$ form a sequence of dependent random variables with a complicated distribution, except for the two previously mentioned processes, see Pitman (1996) for details.

\begin{gather*}
T \sim \gamma \\
\tilde{J}_1|T \sim \text{SBS}(T) \\
\tilde{J}_2|T,\tilde{J}_1 \sim \text{SBS}(T - \tilde{J}_1) \\
\vdots \\
\tilde{J}_{l}|T,\tilde{J}_1,\dots,\tilde{J}_{l-1} \sim \text{SBS}(T - \sum_{i<l} \tilde{J}_i) \\
\vdots \\
\end{gather*}
the corresponding weights are:
$$ \pi_l := \frac{\tilde{J}_{l}}{T - \sum_{i<l} \tilde{J}_i} $$


Cf meeting with Ben
$\beta_i = \frac{\tilde{J}_i}{T_{i-1}}$
$\tilde{p_i} = \beta_i \prod_{j=1}^i (1-\beta_i)$


\subsection{PG}
cf \cite{James:2013uk}

\section{Models of interest}
calculus for SMC for BNP mixture with fixed parameters
fixed parameters are not a assumption, since can be made random then with PMCMC

write model as streaming/online to avoid IS and take advantage of resampling
as state-space model

PPL's code ?

\section{Sampler}
Now that we have detailed and implemented a generative process (i.e. sampling from the prior) for our class of models, we focus in this section on the sampler scheme to use within the \gls{PPL}.

\textcolor{red}{paper reference for particle methods for high dimensional space}
It has been shown that particle methods are the class of sampler scheme to use in the case of high dimensional space.
We are interested in the more general Bayesian case were the parameters are also random variables and to be inferred. Consequently, \acrlong{PMCMC} are methods of choice.

Yet, there is a big path degeneracy issue.

PG -> path degeneracy
PPMH -> ?
PGAS / IPMCMCM


\section{Open questions}
Some questions are still open on this subject, and we are currently working on these.

Is a stick-breaking-like Markov Chain necessary and sufficient for doing inference with particle methods?


What should be the representation of the state in the PPL (theoretically and efficiently concerned) ? Unique components + assignments | all components | sticks lengths ? 

Other models than infinite mixture models  



\chapter{Future Work}
In this chapter we present several ideas which could be further developed. When matured enough, this ideas may become projects on themselves.


\section{Learning parameters in PPL}
\subsection{Motivation}
Since variational methods have arisen, ideas of mixing sampling with \gls{VI} have emerged, including in the \gls{PPL} literature.

In \cite{Wingate:2013tq}, the authors introduce the idea of automatically learning the parameters of proposals for \gls{SMC} within a \gls{PPL}. A lower bound on the \emph{KL} divergence between the proposal and the true posterior distribution is optimize via gradient descent.
In \cite{Ritchie:2016ve, Le:2016te}, this idea is further developed using neural networks (such as LSTMs \cite{Hochreiter:1997fq}) to parametrize these proposal distributions. These networks are fed with the previous latent and observed variables.
In AESMC \cite{Le:2017wm}, FIVO \cite{Maddison:2017wp} and VSMC \cite{Naesseth:2017tl}, both the model and the \gls{SMC}'s proposal are learned by maximizing the marginal likelihood estimator given by the \gls{SMC}.

The interest in learning parameters (for proposals and for the model) and performing inference on some random variables at once is thus great. \gls{PPL}s allow to easily write probabilistic models and perform inference on latent variables. One may be interested in building a \gls{PPL} with the capability of automatically optimizing some parameters given a loss/estimator.

\gls{AD} methods \cite{Baydin:2015up} enable the computation of gradients of some variables with respect to some parameters. The reverse differentiation is particularly popular in the machine learning community, where the history of each variable (how it has been constructed) is saved as a computational graph, and gradients can then be computed via the \textit{chain rule}.

Some libraries such as TensorFlow \cite{Tensorflow} require the users to define static computation graphs within the syntactic and semantic constraints of a domain-specific mini language with limited support for control flow whereas the lineage of projects leading from autograd \footnote{\url{https://github.com/HIPS/autograd}} to PyTorch \footnote{\url{http://pytorch.org/}} provide truly general-purpose reverse mode \gls{AD} capability. The latter mode is to be preferred for fully and easy support of control flow such as stochastic recursion which is needed for stick-breaking processes.


\subsection{Choice of language/library}
We this idea in mind, one can now think how to pragmatically build such a \gls{AD} \gls{PPL}.

\paragraph{Python:}
One of the most famous language for scientific computing is Python \cite{Rossum:1995:PRM:869369}.
As \emph{Edward} \cite{Edward} is built on top of \emph{Tensorflow}, one could build a \gls{PPL} layer on top of PyTorch. \emph{Edward} implements each \gls{MCMC} step (specific for each algorithms) as a computational graph in \emph{Tensorflow} which is thereafter run with the updated input so as to sample a new value. Similarly for \gls{VI}, \emph{Edward} implements a loss function as a computational graph, for which its gradient can be computed via auto-differentiation.

However, \emph{Edward} focuses on \gls{VI} and \gls{HMC}-like schemes and does not handle particle algorithms.
Indeed, so as to handle such algorithms, a \gls{PPL} must have access to \textit{breakpoints} at \texttt{assume} statement. This can be implemented via \gls{CPS} \footnote{\url{http://matt.might.net/articles/by-example-continuation-passing-style/}} or coroutine \footnote{\url{https://en.wikipedia.org/wiki/Coroutine}} copying. Unfortunately implementing \gls{CPS} is something non-trivial.

\paragraph{Julia:}
One could also think of using Julia \cite{Bezanson:2017gd}, which has been specifically built for scientific usage. Julia has the advantage of natively handling coroutine copying, which is used in Turing \cite{Turing} to implement particle methods.

Reverse mode \gls{AD} libraries exist in Julia, \emph{ReverseDiff.jl} \footnote{\url{https://github.com/JuliaDiff/ReverseDiff.jl}} and \emph{Knet.jl} \footnote{\url{https://github.com/denizyuret/Knet.jl}} which respectively build a static and dynamic graph.

I am particularly interested in the perspective of adapting a \gls{AD} library for Turing \cite{Turing}.

\subsection{New models}
With such as \gls{PPL} in mind, one can think of new model or algorithms to be developed.

The idea of AESMC \cite{Le:2017wm} might be extended to \gls{PG} and \gls{PMMH} so as to learn proposals' parameters for their \gls{SMC} and for $p(\theta^\star|\theta)$ parameters (specific of \gls{PMMH}).

\subsection{Difficulties}
Yet this is not a trivial task, one have to put proper care when computing unbiased gradient of a loss function defined by an expectation over a collection of random variables.

Hopefully, a stochastic computation graph \cite{Schulman:2015wk} can be converted into a deterministic computation graph, to which the backpropagation algorithm can then be applied on a surrogate loss function which results in an unbiased gradient estimator of the loss.


\section{Piecewise Deterministic Markov Processes}
A novel class of non-reversible Markov chain Monte Carlo schemes relying on continuous-time piecewise deterministic Markov Processes has recently emerged \cite{Vanetti:2017ux}. In these algorithms, the state of the Markov process evolves according to a deterministic dynamics which is modified using a Markov transition kernel at random event times. A general framework is presented in \cite{Bierkens:2017we}, and includes among others the Zig-Zag Process \cite{Bierkens:2016uk}, the Bouncy Particle Sampler \cite{BouchardCote:2017gs} and the Generalized Bouncy Particle Sampler \cite{Wu:2017uz}.


It has been claimed \cite{Bierkens:2017we} that the non-reversibility property of these algorithms enhances the mixing rate of the chain.
I am consequently interested in understanding to what extent this class of MCMC schemes could fit the \gls{PPL}'s setting.


\section{Variational Inference for \gls{BNP} in \gls{PPL}} \label{BNP_VI}
As explained in Section \ref{BNP_VI}, Truncation-free variational inference methods rely on a lazy representing of the clusters assignments. Yet, the marginalisation used seems to be only available for few models.

However, we may be able to use a similar approach for more flexible \gls{BNP} models, by extending the latent space with the sticks proportions and mixture components (since they cannot be marginalized out).
Moreover, there might be a deeper link between \textit{Truncation-free} \gls{VI} and stick-breaking processes.



\section{Adversarial Inference for \gls{BNP} in \gls{PPL}}
Adversarial inference methods \cite{Dumoulin:2016td, Donahue:2016wo, DBLP:conf/icml/MeschederNG17} inspired by GANs \cite{Goodfellow:2014wp} jointly learns a generation network and an inference network using an adversarial process.

The decoder/generator network $x' \sim p(x|z)$ maps samples from stochastic latent variables to the data space while the encoder/inference network $z' \sim q(z|x)$ maps training examples in data space to the space of latent variables.

An adversarial game is cast between these two networks and a discriminative network is trained to distinguish between joint latent/data-space samples $(x', z)$ from the generative network and joint samples $(x, z')$ from the inference network.

Adversarial inference seems to be closely related to \gls{VI}. Yet, in adversarial inference the model can also be learned as opposed to \gls{VI} where only the proposal is learned. Moreover, in \gls{VI} the marginal likelihood is optimised via a lower bound (ELBO) whereas in adversarial inference, a classification loss is optimised.

This approach could be interesting in the \gls{BNP} setting, if a tractable and tight lower bound on the marginal likelihood cannot be found.



%!TEX root = internshipReport.tex

\chapter{Conclusion and personal review}

\section{Conclusion}
After giving some background on the mission, we reviewed well-known \acrlong{BNP} models, with a focus on infinite mixture models and discrete random probability measures. After we reviewed the design of \acrlongpl{PPL} and described the framework in which inference schemes can be generically applied for these programs.
Then we recalled a generative construction of \acrlongpl{PKP} called \acrlong{SBS} and stressed out that it relates with a high order \glspl{PPL}, via stochastic recursion.
We implemented such as a generative process for some \gls{BNP} classes, along with a mixture model in a \gls{PPL} named Turing, and ran experiments to assess the performance of posterior samplers.

Further work is still needed to design and implement an effective posterior sampler.
Furthermore, this project could lead to other interesting ideas. Such as performing inference whithin a \gls{PPL} with a Piecewise Deterministic \gls{MCMC} scheme, with variational inference or maybe via adversarial inference. Also, enabling a \gls{PPL} to automatically compute gradients via auto-differentation could lead to new models.


\section{Personal review}
I am much satisfied by this internship which brought me much experience and knowledge.
I worked during four months on a subject involving Bayesian nonparametrics, Computational Statistics and Programing Languages, while my knowledge of these fields was limited. I therefore had to learn a lot about these, obviously by reading articles and books, but also by actually implementing some inference schemes. Talking with other students, post-docs and professors is also a stimulating way to get to know more about other related subjects. I was for instance invited by Frank Wood \footnote{He works on \acrlongpl{PPL}, \url{http://www.robots.ox.ac.uk/~fwood/}} in the Department of Engineering \footnote{\url{http://www.eng.ox.ac.uk}} to have a talk with him and some of his students. Reading groups are good places to frequently meet with the other students and keep up with the state of the art. I have been leading the probabilistic inference reading group \footnote{\url{https://github.com/BigBayes/oxsml/wiki/Probabilistic-Inference-meetings}} and will also participate to the Reinforcement Learning reading group \footnote{\url{http://www.stats.ox.ac.uk/~cmaddis/}} organised by Chris Maddison.

I also have much appreciated the great opportunities brought by this internship/PhD position, such as OxCSML \footnote{Oxford Computational Statistics \& Machine Learning} weekly talks \footnote{\url{http://csml.stats.ox.ac.uk/events/}} or other department's talks. I particularly enjoyed Judith Rousseau's \footnote{\url{https://www.ceremade.dauphine.fr/~rousseau/}} talk on posterior consistency and Jim Pitman's \footnote{\url{https://www.stat.berkeley.edu/~pitman/}} talk on transformations of Brownian processes. I have also been invited for a week to the Microsoft Research AI Summer School 2017 \footnote{\url{https://www.microsoft.com/en-us/research/event/ai-summer-school-2017/}} in Cambridge, where I met many other PhD students working in machine learning related areas.

Moreover, I made lots of progress in my research workflow. Before, coming here, even if I had already worked on several so called \textit{research projects}, I had never actually done any research. I am still at the beginning of my research ``career'', yet I believe to better understand the spirit of research; iterating back and forth between questions and answers, by experimenting, reading, writing and discussing with others.
I have learned to enjoy and take the most of the freedom and autonomy I possess by being a PhD student, which was quite destabilizing at first.


%\backmatter

% \appendix
% \chapter*{Appendices}
% \addcontentsline{toc}{chapter}{Appendices}
% \renewcommand{\thesection}{\Alph{section}}

% \section{Variational Inference ?}
% \section{Automatic Differentiation ?}
% \section{Exchangeability} \label{exchangeability}
% The underlying assumption of all Bayesian methods is that the parameter specifying the observation model is a random variable. This assumption is subject to much criticism, and at the heart of the Bayesian versus non-Bayesian debate that has long divided the statistics community. However, there is a very general type of observations for which the existence of such a random variable can be derived mathematically: For so-called \textit{exchangeable} observations, the Bayesian assumption that a randomly distributed parameter exists is not a modeling assumption, but a mathematical consequence of the data’s properties.

% Formally, a sequence of variables $X_1,X_2,\dots,X_n$ over the same probability space $(\mathcal{X},\Omega)$ is \textit{exchangeable} if their joint distribution is invariant to permuting the variables. That is, if $P$ is the joint distribution and $\sigma$ any permutation of $\{1,\dots,n\}$, then
% $$ P(X_1=x_1,\dots,X_n=x_n) = P(X_1=x_{\sigma(1)},\dots,X_n=x_{\sigma(n)}) $$

% An infinite sequence $X_1,X_2,\dots$ is \textit{infinitely exchangeable} if $X_1,\dots,X_n$ is \textit{exchangeable} for every $n \ge 1$. Exchangeability reflects the assumption that the variables do not depend on their indices although they may be dependent among themselves. This is typically a reasonable assumption in machine learning and statistical applications, even if the variables are not themselves iid (independently and identically distributed). Exchangeability is a much weaker assumption than iid since iid variables are automatically exchangeable.

% If $\theta$ parametrizes the underlying distribution, and one assumes a prior distribution over $\theta$, then the resulting marginal distribution over $X_1,X_2,\dots,X_n$ with $\theta$ marginalized out will still be exchangeable. A fundamental result credited to de Finetti \cite{finetti31} states that the converse is also true. That is, if $X_1,X_2,\dots,X_n$ is (infinitely) exchangeable, then there is a random $\theta$ such that:
% \begin{equation} \label{eq:deFineti}
% P(X_1,\dots,X_n) = \int{P(\theta)\prod_{i=1}^n{P(X_i|\theta) d\theta}}
% \end{equation}
% for every $n \ge 1$. In other words, the seemingly innocuous assumption of exchangeability automatically implies the existence of a hierarchical Bayesian model with $\theta$ being the random latent parameter. This the crux of the fundamental importance of exchangeability to Bayesian statistics.

% In de Finetti’s Theorem it is important to stress that $\theta$ can be infinite dimensional (it is typically a random measure), thus the hierarchical Bayesian model \ref{eq:deFineti} is typically a nonparametric one. For example, the Blackwell-MacQueen urn scheme (related to the \gls{CRP}) is exchangeable thus implicitly defines a random measure, namely the \gls{DP}.

%\nocite{*}

\bibliographystyle{siam}
\bibliography{rapport}

\end{document}
