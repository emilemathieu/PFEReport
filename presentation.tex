\chapter{Presentation of the Department of Statistics}


\section{Creation}
\quad The Department of Statistics \footnote{\url{https://www.stats.ox.ac.uk}} is part of the University of Oxford, along with the other departments and the 38 constituent colleges.
The University of Oxford was founded in the 11th century, which makes it the oldest university in the English-speaking world and the world's second-oldest university in continuous operation.

The Department of Statistics was officially created in 1988, even though first moves in the development of Oxford statistics can be dated to the 19th century.

Indeed, In the 1870s, Florence Nightingale -- the pioneer of modern nursing -- discussed the possibility of endowing a Professorship of Statistics in Oxford, but the proposal eventually foundered.
However, Oxford did appoint a statistician to a chair in 1891, although not to a chair in statistics.

The next significant moves in the development of Oxford statistics were by economists, who were increasingly keen to build economic theory on a foundation of sound data analysis.
This led to the creation in 1935 of an Institute of Statistic, swhich was then renamed as the Institute of Economics and Statistics in 1962. 

The sequence of events which led directly to the establishment of the present Department of Statistics began with the appointment in 1945 of David Finney as the university’s first Lecturer in the Design and Analysis of Scientific Experiment (LIDASE).

Then in the 1980s, after the Department of Biomathematics' head increasingly felt that Oxford was losing out in the face of developments in statistics, a working party appointed by the general board of the university to assess a careful analysis of the organisation of statistics in Oxford. 
They found fragmentation to be ‘the dominant feature of Oxford statistics’ and concluded that ‘fragmentation has serious disadvantages ....’.
The working party’s report recommended the creation of a university statistics department, which were to include the former Department of Biomathematics, together with a new Professorship in Statistical Science and the two existing lecturerships in statistics within the Mathematical Institute.

These major recommendations were all accepted by the university and the new Department of Statistics was created in 1988.

\section{Activities}
\quad The Department of Statistics at Oxford is a world leader in research including computational statistics and statistical methodology, applied probability, bioinformatics and mathematical genetics.
The main research groups in the Department are Computational statistics and machine learning, Probability, Statistical genetics and bioinformatics, Protein Informatics and  Statistical Genetics.

I am part of the Computational Statistics and Machine Learning Group (OxCSML) \footnote{\url{http://csml.stats.ox.ac.uk/people/mathieu/}}, which have research interests spanning Statistical Machine Learning, Monte Carlo Methods and Computational Statistics, Statistical Methodology and Applied Statistics. \\

The department offers an undergraduate degree (BA or MMath) in Mathematics and Statistics, jointly with the Mathematical Institute. 
At postgraduate level there is an MSc course in Applied Statistics (MSc in Statistical Science from 2017), as well as a lively and stimulating environment for postgraduate research (DPhil or MSc by Research).
The department also has a consulting activity called \textit{Oxford University Statistical Consulting}.

%\section{Atmosophere} ?
% Department day, brunch, lunches, pub
%Reading Group ?
