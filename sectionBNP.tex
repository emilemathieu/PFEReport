\chapter{Bayesian nonparametric}

\section{Definition} \label{BNP_def}

\section{Usefulness}

\section{Canonical models}
???
DP, PYP, etc

\section{MCMC Inference}
% Variational inference ? DPM \cite{DPVI}

\quad Constructing MCMC schemes for models with one or more Bayesian nonparametric components is an active research area since dealing with the infinite dimensional component $P$ forbids the direct use of standard simulation-based methods.These methods usually require a finite-dimensional representation. The general idea for designing inference schemes is to find finite dimensional representations to be able to store the model in a computer with finite capacity.

There are two main sampling approaches to facilitate simulation in the case of Bayesian nonparametric models: random truncation and marginalisation. These two schemes are known in the literature as conditional and marginal samplers.

\subsection{Marginal Samplers}
\quad Marginal samplers bypass the need to represent the infinite-dimensional component by marginalising it out. These schemes have lower storage requirements than conditional samplers because they only store the induced partition, but could potentially have worse mixing properties.

\subsection{Conditional Samplers}
\quad Conditional samplers replace the infinite-dimensional prior by a finite-dimensional representation chosen according to a truncation level. Since these samplers do not integrate out the infinite-dimensional component, their output provides a more comprehensive representation of the random probability measure.
thinning vs stick-breaking

\subsection{Hybrid Samplers}
YW paper on PK ?

\subsection{SMC}
Review of SMC ?
cf Maria Lomeli thesis
